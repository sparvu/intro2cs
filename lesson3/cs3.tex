\documentclass[aspectratio=1610]{beamer}
\usetheme{boxes}
\usecolortheme{crane}
\usepackage{amsmath,amsfonts}
\usepackage{algpseudocode}
\usepackage{multicol}
\usepackage{pgfplots}
\pgfplotsset{compat=1.15}
\usepackage{mathrsfs}
\usetikzlibrary{arrows}


%-------------------------------------------------------------------
%	 TITLE SLIDE
%-------------------------------------------------------------------


\begin{document}

% -------------------------------------------------------------------
% Lesson 3
% -------------------------------------------------------------------
\section{Data Structures and Algorithms}

\begin{frame}
\begin{center}
\Huge Lesson 3\\~\\
\textbf{Data Structures and Algorithms DSA}
\end{center}
\end{frame}



\begin{frame}
\frametitle{Lesson 3}

\Huge In this lesson we will talk about:
 \alert{data structures, analysing and designing algorithms, algorithm efficiency, searching and sorting algorithms}
\end{frame}



\begin{frame}{Lesson 3}{}
\begin{center}
\Huge \textbf{Data Structures}
\end{center}
\end{frame}


\begin{frame}{Lesson 3}{}
\Huge{What is a data structure?}\\~\\
\includegraphics[scale=0.80]{Images/ds}
\end{frame}



\begin{frame}{Lesson 3}{}
\LARGE
\textbf{Data Structures}\\~\\
\begin{itemize}
    \item How to \textbf{organize} data
    \item For \textbf{efficient} access
\end{itemize}

Its a collection of data values, and the relationships among these values
\end{frame}


\begin{frame}{Lesson 3}{}
\LARGE
\textbf{Sets}\\~\\
The basic, fundamental data structure: \{1,2,4,51,9\}
\begin{itemize}
    \item mathematical set
    \item unchanging
    \item contains a fixed number of elements
\end{itemize}

\end{frame}



\begin{frame}{Lesson 3}{}
\LARGE
\textbf{Sets}\\~\\
As mathematical sets are unchanging, sets which are manipulated by algorithms are dynamic. Can change in size, grow or shrink, basically change over the time. 

\begin{itemize}
    \item static - 5 elements \{1,2,4,51,9\}
    \item dynamic - can add or remove elements
\end{itemize}

\end{frame}



\begin{frame}{Lesson 3}{}
\Huge{But what is a set?}\\~\\
\end{frame}


\begin{frame}{Lesson 3}{}
\LARGE
\textbf{Sets}\\~\\
\begin{itemize}
    \item The empty set \{\}
    \item Natural numbers: $\mathbb{N} = \{0, 1, 2, 3, \ldots\}$
    \item Natural numbers except 0: $\mathbb{N^*} = \{1, 2, 3, \ldots\}$
    \item Integers: $\mathbb{Z} = \{\ldots, -3, -2, -1, 0, 1, 2, 3, \ldots\}$
    \item Positive integers: $\mathbb{Z_+} = \{0, 1, 2, 3, \ldots\}$
\end{itemize}
\end{frame}


\begin{frame}{Lesson 3}{}
\LARGE
\textbf{Basic Operations on Sets}\\~\\
\begin{itemize}
    \item insert
    \item delete
    \item test if the element belongs to a set or not 
\end{itemize}

A dynamic set which supports all these basic operations is a dictionary
\end{frame}



\begin{frame}{Lesson 3}{}
\begin{center}
\includegraphics[scale=0.45]{Images/numbers3}
\end{center}
\end{frame}


\begin{frame}{Lesson 3}{}
\begin{center}
\includegraphics[scale=0.14]{Images/NZQRC}
\end{center}
\end{frame}



\begin{frame}{Lesson 3}{}
\LARGE
\textbf{Data structures}\\~\\
\begin{itemize}
    \item Arrays
    \item Matrices
    \item Stacks
    \item Queues
    \item Linked lists
    \item Trees
\end{itemize}
\end{frame}


\begin{frame}{Lesson 3}{}
\begin{center}
\Huge Arrays
\end{center}
\end{frame}

\begin{frame}{Lesson 3}{}
\LARGE
\textbf{Arrays}
\begin{center}
\includegraphics[scale=0.65]{Images/arrays}
\end{center}
\end{frame}


\begin{frame}{Lesson 3}{}
\begin{center}
\includegraphics[scale=0.17]{Images/array2.png}
\end{center}
\end{frame}



\begin{frame}{Lesson 3}{Arrays}
\Large
\textbf{Example: Find the max value in the array A}\\~\\


\label{MaxArray}
\begin{algorithmic}[1]
\Procedure{MaxArray}{$A$} \Comment{Returns the max value in A}
\State $L\gets length(A)$
\State $MAX\gets A[0]$
\For{\texttt{i=1 to L-1}}
\If {$A[i] > MAX$}
    \State \textbf{$MAX = A[i]$} \Comment{The MAX is A[i]}
\EndIf
\EndFor
\State \Return $MAX$
\EndProcedure
\end{algorithmic}

\end{frame}



\end{document}

