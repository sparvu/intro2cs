\documentclass[aspectratio=1610]{beamer}
\usetheme{boxes}
\usecolortheme{crane}
\usepackage{amsmath,amsfonts}
\usepackage{algpseudocode}
\usepackage{multicol}
\usepackage{pgfplots}
\pgfplotsset{compat=1.15}
\usepackage{mathrsfs}
\usetikzlibrary{arrows}


%-------------------------------------------------------------------
%	 TITLE SLIDE
%-------------------------------------------------------------------


\begin{document}

% -------------------------------------------------------------------
% Lesson 3
% -------------------------------------------------------------------
\section{Data Structures and Algorithms}

\begin{frame}
\begin{center}
\Huge Lesson 3\\~\\
\textbf{Data Structures and Algorithms DSA}
\end{center}
\end{frame}



\begin{frame}
\frametitle{Lesson 3}

\Huge In this lesson we will talk about:
 \alert{data structures, analysing and designing algorithms, algorithm efficiency, searching and sorting algorithms}
\end{frame}



\begin{frame}{Lesson 3}{}
\begin{center}
\Huge \textbf{Data Structures}
\end{center}
\end{frame}


\begin{frame}{Lesson 3}{}
\Huge{What is a data structure?}\\~\\
\includegraphics[scale=0.80]{Images/ds}
\end{frame}


\begin{frame}{Lesson 3}{}
\LARGE
\textbf{Data Structures}\\~\\
\begin{itemize}
    \item How to \textbf{organize} data
    \item For \textbf{efficient} access
\end{itemize}

Its a collection of data values, and the relationships among these values
\end{frame}


\begin{frame}{Lesson 3}{}
\LARGE
\textbf{Data structures}\\~\\
\begin{itemize}
    \item Sets
    \item Arrays
    \item Matrices
    \item Stacks
    \item Queues
    \item Linked lists
    \item Trees
\end{itemize}
\end{frame}


\begin{frame}{Lesson 3}{}
\begin{center}
\Huge Sets 
\end{center}
\end{frame}


\begin{frame}{Lesson 3}{}
\LARGE
\textbf{Sets}\\~\\
The basic, fundamental data structure: \{1,2,4,51,9\}
\begin{itemize}
    \item mathematical set
    \item unchanging, unique elements, no duplicates
    \item contains a fixed number of elements or infinite
\end{itemize}

\end{frame}


\begin{frame}{Lesson 3}{}
\Huge{A set of polygons}
\begin{center}
\includegraphics[scale=0.10]{Images/set}
\end{center}
\end{frame}


\begin{frame}{Lesson 3}{}
\LARGE
\textbf{Sets}\\~\\
A set is fixed, not changing. Sets which are manipulated by algorithms are dynamic. These sets can change in size, grow or shrink, basically change over the time. 
\begin{itemize}
    \item static set - \{1,2,4,51,9\}
    \item dynamic set - can add or remove elements
\end{itemize}

\end{frame}



\begin{frame}{Lesson 3}{}
\Huge{But what is a set?}\\~\\
\end{frame}


\begin{frame}{Lesson 3}{}
\LARGE
\textbf{Sets}\\~\\
A set is a \textbf{mathematical model} for a collection of different things, a set contains elements or members, which can be mathematical objects of any kind numbers, symbols, points in space, lines, other geometrical shapes, variables, or even other sets.

\end{frame}


\begin{frame}{Lesson 3}{}
\LARGE
\textbf{Sets}\\~\\
\begin{itemize}
	\item \{white, blue, red, yellow\}
    \item The empty set \{\}
    \item Natural numbers: $\mathbb{N} = \{0, 1, 2, 3, \ldots\}$
    \item Natural numbers except 0: $\mathbb{N^*} = \{1, 2, 3, \ldots\}$
    \item Integers: $\mathbb{Z} = \{\ldots, -3, -2, -1, 0, 1, 2, 3, \ldots\}$
    \item Positive integers: $\mathbb{Z_+} = \{0, 1, 2, 3, \ldots\}$
\end{itemize}
\end{frame}



\begin{frame}{Lesson 3}{}
\Huge{\textbf{\{1,2,3,4\}}}\\~\\
\LARGE
Roster or the enumeration notation defines a set by listing its elements between curly brackets, separated by commas.
\end{frame}


\begin{frame}{Lesson 3}{}
\LARGE
\textbf{Basic Operations on Sets}\\~\\
\begin{itemize}
    \item Insert
    \item Delete
    \item Test - if element X belongs to a set or not 
\end{itemize}

A dynamic set which supports all these basic operations: \textbf{a dictionary} 
\end{frame}


\begin{frame}{Lesson 3}{}
\LARGE
\textbf{Advantages}\\~\\
Perform operations on a collection of elements in an \textbf{efficient} and \textbf{organized} way.
\end{frame}


\begin{frame}{Lesson 3}{}
\begin{center}
\includegraphics[scale=0.45]{Images/numbers3}
\end{center}
\end{frame}


\begin{frame}{Lesson 3}{}
\begin{center}
\includegraphics[scale=0.14]{Images/NZQRC}
\end{center}
\end{frame}


\begin{frame}{Lesson 3}{}
\LARGE
\textbf{Sets: Conclusions}\\~\\
\begin{itemize}
    \item unique elements
    \item no duplicates
    \item unchanging 
    \item fixed or infinite number of elements
\end{itemize}

\end{frame}


\begin{frame}{Lesson 3}{}
\LARGE
\textbf{Im confused. Is a data structure set like a Python set? Or what is the difference?}\\~\\
\end{frame}



\begin{frame}{Lesson 3}{}
\begin{center}
\Huge Arrays
\end{center}
\end{frame}

\begin{frame}{Lesson 3}{}
\LARGE
\textbf{Arrays}
\begin{center}
\includegraphics[scale=0.65]{Images/arrays}
\end{center}
\end{frame}


\begin{frame}{Lesson 3}{}
\begin{center}
includegraphics[scale=0.17]{Images/array2.png}
\end{center}
\end{frame}



\begin{frame}{Lesson 3}{Arrays}
\Large
\textbf{Example 1: Traversing the array A}\\~\\


\label{getArray}
\begin{algorithmic}[1]
\Procedure{getArray}{$A$} \Comment{Returns the max value in A}
\State $L\gets length(A)$
\For{\texttt{i=0 to L-1}}
    \State print $A[i]$
\EndFor
\EndProcedure
\end{algorithmic}
\end{frame}


\begin{frame}{Lesson 3}{Arrays}
\Large
\textbf{Example 2: Find the max value in the array A}\\~\\


\label{MaxArray}
\begin{algorithmic}[1]
\Procedure{MaxArray}{$A$} \Comment{Returns the max value in A}
\State $N\gets length(A)$
\State $MAX\gets A[0]$
\For{\texttt{from i=1 to N-1}}
\If {$A[i] > MAX$}
    \State \textbf{$MAX = A[i]$} \Comment{The MAX is A[i]}
\EndIf
\EndFor
\State \Return $MAX$
\EndProcedure
\end{algorithmic}
\end{frame}



\begin{frame}{Lesson 3}{Arrays}
\Large
\textbf{Example 3: Search element X in array A}\\~\\
\label{SearchArray}
\begin{algorithmic}[1]
\Procedure{SearchArray}{$A$} \Comment{Returns the max value in A}
\State $X\gets MyElement$
\State $N\gets length(A)$
\For{\texttt{from i=0 to N-1}}
\If {$X = A[i]$}
    \State \Return {$i$}  \Comment{The index for my match}
\EndIf
\EndFor
\State \Return -1 \Comment{otherwise return -1}
\EndProcedure
\end{algorithmic}
\end{frame}


\begin{frame}{Lesson 3}{}
\begin{center}
\Huge Matrices
\end{center}
\end{frame}


\begin{frame}{Lesson 3}{}
\LARGE
\textbf{Matrices}
\begin{center}
\includegraphics[scale=0.65]{Images/matrices}
\end{center}
\end{frame}


\begin{frame}{Lesson 3}{}
\LARGE
\textbf{Matrices}
\begin{center}
\includegraphics[scale=0.14]{Images/matrices2}
\end{center}
\end{frame}


\end{document}

