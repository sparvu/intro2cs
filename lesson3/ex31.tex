
\documentclass[aspectratio=1610]{beamer}
\usetheme{boxes}
\usecolortheme{crane}
\usepackage{amsmath,amsfonts}
\usepackage{algpseudocode}
\usepackage{multicol}



\begin{document}



% --------------------------------------------------------------
% Exercise 1 
% --------------------------------------------------------------
\begin{frame}{Lesson 3}{}
\begin{center}
\Huge Exercise 3.1
\end{center}
\end{frame}

\begin{frame}{Lesson 3, Exercise 3.1}{}
\Large
\textbf{3.1.1 Please answer the following questions:}\\
\begin{align}
   \begin{bmatrix}
        1 & 2 & 3\\
        4 & 5 & 6\\
        7 & 8 & 9
     \end{bmatrix}
     \times
     \begin{bmatrix}
       9 & 8 & 7\\
       6 & 5 & 4\\
       3 & 2 & 1
     \end{bmatrix} =
\end{align}   
\begin{align}
   \begin{bmatrix}
        1 & 2 & 3\\
        4 & 5 & 6\\
        7 & 8 & 9
     \end{bmatrix}
     \times
     5 =
\end{align}

\end{frame}


\begin{frame}{Lesson 3, Exercise 3.1}{}
\Large
\textbf{3.1.2 Design the algorithm to solve:}\\~\\
Write a recursive function which reads as input a word character with
character, prints the word on standard output, and then prints back to screen the inverse of the word. Mark the end of the word with space. (Pseudocode, Python) 
\\~\\
Return in writing the answers by email, in text format or PDF
\end{frame}


\end{document}

