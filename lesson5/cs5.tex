\documentclass[aspectratio=1610]{beamer}
\usetheme{boxes}
\usecolortheme{crane}
\usepackage{amsmath,amsfonts}
\usepackage{algpseudocode}
\usepackage{multicol}
\usepackage{pgfplots}
\pgfplotsset{compat=1.15}
\usepackage{mathrsfs}
\usepackage{listings}
\lstset{
  columns=flexible
}
\usetikzlibrary{arrows}


%-------------------------------------------------------------------
%	 TITLE SLIDE
%-------------------------------------------------------------------


\begin{document}

% -------------------------------------------------------------------
% Lesson 5
% -------------------------------------------------------------------
\section{Data Pipelines}

\begin{frame}
\begin{center}
\Huge Lesson 5\\~\\
\textbf{Data Pipelines}
\end{center}
\end{frame}


\begin{frame}
\frametitle{Lesson 5}
\Huge In this lesson we will talk about:\\
\huge
 \alert{Data pipelines. Origins and Implementations}\\
 \alert{Data Capture, Transforming \& Analysing}\\
 \alert{Raw Data. Data Aggregation}\\
 \alert{Pipeline efficiency}\\
 \alert{Basic statistical functions}
\end{frame}


\begin{frame}{Lesson 5}{Data Pipelines}
\Huge
\begin{center}
Origins
\end{center}
\end{frame}


\begin{frame}{Lesson 5}{Data Pipelines}
\Huge
\begin{center}
2025. Present time 
\end{center}
\end{frame}



\begin{frame}{Lesson 5}{Data Pipelines}
\Huge
\begin{center}
 IBM
\end{center}
\end{frame}

\begin{frame}{Lesson 5}{Data Pipelines}
\LARGE
"A data pipeline is a method in which raw data is ingested from 
various data sources, transformed and then ported to a data store, 
such as a data lake or data warehouse, for analysis."
\end{frame}

\begin{frame}{Lesson 5}{Data Pipelines}
\LARGE
"Data pipelines act as the “piping” for data science projects or 
business intelligence dashboards. Data can be sourced through a wide 
variety of places: APIs, SQL and NoSQL databases, files."
\end{frame}

\begin{frame}{Lesson 5}{Data Pipelines}
\LARGE
"During sourcing, data lineage is tracked to document the 
relationship between enterprise data in various business and IT 
applications, for example, where data is currently and how it’s 
stored in an environment, such as on-premises, in a data lake or in a
data warehouse."\\~\\
Source: ibm.com
\end{frame}

\begin{frame}{Lesson 5}{Data Pipelines}
\Huge
\begin{center}
www.geeksforgeeks.org
\end{center}
\end{frame}


\begin{frame}{Lesson 5}{Data Pipelines}
\LARGE
"Data Pipeline deals with information that is flowing from one end to 
another. In simple words, we can say collecting the data from various 
resources than processing it as per requirement and transferring it 
to the destination by following some sequential activities."\\~\\
Source: www.geeksforgeeks.org
\end{frame}

\begin{frame}[plain,noframenumbering]
\makebox[\linewidth]{\includegraphics[width=\paperwidth]{Images/pipeline1}}
\end{frame}


\begin{frame}{Lesson 5}{Data Pipelines}
\Huge
\begin{center}
AMAZON
\end{center}
\end{frame}


\begin{frame}{Lesson 5}{Data Pipelines}
\LARGE
"A data pipeline is a series of processing steps to prepare 
enterprise data for analysis. Organizations have a large volume of 
data from various sources like applications, Internet of Things (IoT)
devices, and other digital channels. However, raw data is useless; it
must be moved, sorted, filtered, reformatted, and analyzed for
business intelligence. A data pipeline includes various technologies
to verify, summarize, and find patterns in data to inform business 
decisions."
\end{frame}


\begin{frame}{Lesson 5}{Data Pipelines}
\LARGE
"Just like a water pipeline moves water from the reservoir to your
taps, a data pipeline moves data from the collection point to
storage. A data pipeline extracts data from a source, makes changes,
then saves it in a specific destination. We explain the critical
components of data pipeline architecture below."
\end{frame}


\begin{frame}[plain,noframenumbering]
\makebox[\linewidth]{\includegraphics[width=\paperwidth]{Images/amazon_pipeline}}
\end{frame}


\begin{frame}{Lesson 5}{Data Pipelines}
\Huge
 Data pipeline = a program or what? Lets rollback time... 
 \end{frame}


\begin{frame}{Lesson 5}{Data Pipelines}
\Huge
\begin{center}
1969. UNIX 
\end{center}
\end{frame}

\begin{frame}{Lesson 5}{Data Pipelines}
\LARGE
\textbf{The very first data pipelines}\\~\\
As early as 1964 Douglas McIlroy thought about a mechanism how we 
could connect programs same way like we connect and combine
garden hoses together. 
\end{frame}


\begin{frame}[plain,noframenumbering]
\makebox[\linewidth]{\includegraphics[width=\paperwidth]{Images/garden-pipeline}}
\end{frame}


\begin{frame}[plain,noframenumbering]
\makebox[\linewidth]{\includegraphics[width=\paperwidth]{Images/garden-pipeline2}}
\end{frame}


\begin{frame}[plain,noframenumbering]
\makebox[\linewidth]{\includegraphics[width=\paperwidth]{Images/garden-pipeline3}}
\end{frame}

\begin{frame}{Lesson 5}{Data Pipelines}
\LARGE
\textbf{Connecting Garden Hoses}\\~\\
\begin{itemize}
    \item it should connect different hoses
    \item as efficiently as possible
    \item with no leaks
    \item covering all the garden
\end{itemize}
\end{frame}



\begin{frame}{Lesson 5}{Data Pipelines}
\LARGE
\textbf{Pipelines are everywhere}\\~\\
\begin{itemize}
    \item transport fresh water pipelines
    \item collection pipelines from ground gas and oil to refineries
    \item wastewater pipelines
    \item distribution pipelines
\end{itemize}
\end{frame}


\begin{frame}[plain,noframenumbering]
\makebox[\linewidth]{\includegraphics[width=\paperwidth]{Images/water-pipeline}}
\end{frame}



\begin{frame}{Lesson 5}{Data Pipelines}
\Huge
 The idea was to reuse the model from other industries to CS
 \end{frame}


\begin{frame}
\begin{center}
\Huge
\begin{quote}
\textbf{"A pipeline is a mechanism for connecting the output of one program directly and conveniently into the 
    input of another program."}
\begin{flushright}
{--- Brian Kernighan, UNIX father}	
\end{flushright}
\end{quote}
\end{center}
\end{frame}


\begin{frame}{Lesson 5}{Data Pipelines}
\Huge
 The very first models of data pipelines were defined, introduced and 
 implemented in the operating systems. \textbf{Enter UNIX}
 \end{frame}




\begin{frame}{Lesson 5}{Data Pipelines}
\LARGE
\textbf{UNIX pipelines}\\~\\
The very first data pipeline was implemented for UNIX based operating
systems. The ideas was very simple: combine one or many programs 
using the \textbf{pipe} operator \text{\textbar}
\end{frame}


\begin{frame}{Lesson 5}{Data Pipelines}
\Huge
\begin{center}
\textbf{program} \text{\textbar} \textbf{program} \text{\textbar} ... 
\end{center}
\end{frame}



\begin{frame}[plain,noframenumbering]
\makebox[\linewidth]{\includegraphics[width=\paperwidth]{Images/pipeline}}
\end{frame}


\begin{frame}{Lesson 5}{Data Pipelines}
\Huge
The pipeline: is connecting one or many programs together! It is not 
a single monolithic program.
\end{frame}


\begin{frame}{Lesson 5}{Data Pipelines}
\LARGE
\textbf{UNIX pipelines}\\~\\
\Huge
But why connect different programs instead of having a single big 
program to handle all the input data?
\end{frame}


\begin{frame}{Lesson 5}{Data Pipelines}
\Huge
\begin{center}
Few reasons behind... have small, \textbf{modular} programs which can 
connect together to solve a problem quickly and as modular as
possible
\end{center}
\end{frame}

\begin{frame}{Lesson 5}{Data Pipelines}
\Huge
\begin{center}
And \alert{enforce} a strict \alert{order} of execution!!!
\end{center}
\end{frame}


\begin{frame}{Lesson 5}{Data Pipelines}
\huge
\begin{itemize}
    \item smaller programs are easy to develop 
    \item and are simple and cheaper to maintain
    \item as well the amount of data might not fit one program
    \item no temporary files or data in between
\end{itemize}
\end{frame}


\begin{frame}[plain,noframenumbering]
\makebox[\linewidth]{\includegraphics[width=\paperwidth]{Images/unix-pipeline}}
\end{frame}



\begin{frame}{Lesson 5}{Data Pipelines}
\LARGE
\textbf{UNIX pipelines}\\~\\
Connects different UNIX programs: like ls, sort, wc, grep to solve a problem using the \text{\textbar} pipe character
\end{frame}


\begin{frame}[plain,noframenumbering]
\makebox[\linewidth]{\includegraphics[width=\paperwidth]{Images/unix-pipeline2}}
\end{frame}


\begin{frame}[fragile]
\LARGE
\textbf{UNIX pipelines, example}\\~\\
\begin{lstlisting}[language=sh]
$ ls -lrt Images/*.png | wc -l
     140
\end{lstlisting}
\end{frame}


\begin{frame}[fragile]
\LARGE
\textbf{Sort all images which were produced during Feb, example}\\
\Large
\begin{lstlisting}[language=sh]
$ ls -l Images/ | awk '{print $6,$7,$9}' | grep Feb | sort
Feb 26 treevsgraph.png
Feb 26 treevsgraph2.png
Feb 26 tsp.png
Feb 3 Alan_turing.jpg
Feb 3 Bombe.jpg
Feb 4 Queue.png
...
Feb 4 factorial2.png
Feb 4 recursion.png
\end{lstlisting}
\end{frame}



\begin{frame}[fragile]
\LARGE
\textbf{Sort all images which were produced during Feb, example}\\
\Large
\begin{lstlisting}[language=sh]
$ ls -l Images/ | awk '{print $6,$7,$9}' | grep Feb | sort -V
Feb 19 hash_function4.png
Feb 19 hash_function5.png
Feb 25 binary-tree.png
Feb 26 btree.png
Feb 26 depth-height.png
Feb 26 directed-graph.png
Feb 26 graph.png
Feb 26 selfbalanced-tree.png
Feb 26 treevsgraph.png
...
\end{lstlisting}
\end{frame}


\begin{frame}{Lesson 5}{Data Pipelines}
\Huge
\begin{center}
But how about the pipeline's \alert{speed} and overall
\alert{performance}? Should we care about?
\end{center}
\end{frame}



\begin{frame}[fragile]
\LARGE
\textbf{Search for all png images measuring the elapsed time, example}\\
\Large
\begin{lstlisting}[language=sh]
$ time find $HOME -type f | grep png
...
/myhome/feynman_path/examples/no-entanglement.png
/myhome/feynman_path/examples/no-interference-circuit.png
/myhome/feynman_path/examples/entanglement.png
/myhome/feynman_path/examples/no-entanglement-circuit.png

real 0m9.167s
user 0m0.450s
sys 0m3.147s
\end{lstlisting}
\end{frame}



\begin{frame}[fragile]
\LARGE
\textbf{Search  and sort for all jpg and png images measuring the elapsed time, example}\\
\Large
\begin{lstlisting}[language=sh]
$ time find $HOME -type f | egrep 'png|jpg' | sort -V
...
/myhome/.vscode/extensions/sumneko.lua-3.6.3-darwin-arm64/client/3rd/vscode-lua-doc/doc/en-us/53/osi-certified-72x60.png
/myhome/.vscode/extensions/sumneko.lua-3.6.3-darwin-arm64/client/3rd/vscode-lua-doc/doc/en-us/54/osi-certified-72x60.png
/myhome/.vscode/extensions/sumneko.lua-3.6.3-darwin-arm64/client/3rd/vscode-lua-doc/doc/zh-cn/53/osi-certified-72x60.png
/myhome/.vscode/extensions/sumneko.lua-3.6.3-darwin-arm64/images/logo.png
/myhome/.vscode/extensions/vadimcn.vscode-lldb-1.8.1/images/lldb.png

real 0m9.240s
user 0m0.773s
sys 0m3.034s
\end{lstlisting}
\end{frame}


\begin{frame}{Lesson 5}{Data Pipelines}
\Huge
\begin{center}
Performance matters! Measuring pipeline execution time matters!
Take a look... 
\end{center}
\end{frame}


\begin{frame}[fragile]
\LARGE
\textbf{Search and sort for all jpg and png images measuring the elapsed time under Windows, example}\\
\Large
\begin{lstlisting}[language=sh]
$ time find /myhome -type f | grep png

...
/myhome/.vscode/extensions/sumneko.lua-3.6.3-darwin-arm64/client/3rd/vscode-lua-doc/doc/en-us/53/osi-certified-72x60.png
/myhome/.vscode/extensions/sumneko.lua-3.6.3-darwin-arm64/client/3rd/vscode-lua-doc/doc/en-us/54/osi-certified-72x60.png
/myhome/.vscode/extensions/sumneko.lua-3.6.3-darwin-arm64/client/3rd/vscode-lua-doc/doc/zh-cn/53/osi-certified-72x60.png
/myhome/.vscode/extensions/sumneko.lua-3.6.3-darwin-arm64/images/logo.png
/myhome/.vscode/extensions/vadimcn.vscode-lldb-1.8.1/images/lldb.png


real    2m20.786s
user    0m0.421s
sys     0m1.686s

\end{lstlisting}
\end{frame}




$ time find /cygdrive/c/Users/parvust/ -type f | grep png




\end{document}

