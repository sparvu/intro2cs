\documentclass[aspectratio=1610]{beamer}
\usetheme{boxes}
\usecolortheme{crane}
\usepackage{amsmath,amsfonts}
\usepackage{algpseudocode}
\usepackage{multicol}
\usepackage{pgfplots}
\pgfplotsset{compat=1.15}
\usepackage{mathrsfs}
\usepackage{listings}
\lstset{
  columns=flexible
}
\usetikzlibrary{arrows}


%-------------------------------------------------------------------
%	 TITLE SLIDE
%-------------------------------------------------------------------


\begin{document}

% -------------------------------------------------------------------
% Lesson 5
% -------------------------------------------------------------------
\section{Data Pipelines}

\begin{frame}
\begin{center}
\Huge Lesson 5\\~\\
\textbf{Data Pipelines}
\end{center}
\end{frame}


\begin{frame}
\frametitle{Lesson 5}
\Huge In this lesson we will talk about:\\
\huge
 \alert{Data pipelines. Origins. Types}\\
 \alert{Data Capture, Transforming \& Analysing}\\
 \alert{Raw Data. Data Aggregation}\\
 \alert{Pipeline efficiency}\\
 \alert{Basic statistical functions}
\end{frame}


\begin{frame}{Lesson 5}{Data Pipelines}
\Huge
\begin{center}
Origins
\end{center}
\end{frame}


\begin{frame}{Lesson 5}{Data Pipelines}
\Huge
\begin{center}
2025. Present time 
\end{center}
\end{frame}



\begin{frame}{Lesson 5}{Data Pipelines}
\Huge
\begin{center}
 IBM
\end{center}
\end{frame}

\begin{frame}{Lesson 5}{Data Pipelines}
\LARGE
"A data pipeline is a method in which raw data is ingested from 
various data sources, transformed and then ported to a data store, 
such as a data lake or data warehouse, for analysis."
\end{frame}

\begin{frame}{Lesson 5}{Data Pipelines}
\LARGE
"Data pipelines act as the “piping” for data science projects or 
business intelligence dashboards. Data can be sourced through a wide 
variety of places: APIs, SQL and NoSQL databases, files."
\end{frame}

\begin{frame}{Lesson 5}{Data Pipelines}
\LARGE
"During sourcing, data lineage is tracked to document the 
relationship between enterprise data in various business and IT 
applications, for example, where data is currently and how it’s 
stored in an environment, such as on-premises, in a data lake or in a
data warehouse."\\~\\
Source: ibm.com
\end{frame}

\begin{frame}{Lesson 5}{Data Pipelines}
\Huge
\begin{center}
www.geeksforgeeks.org
\end{center}
\end{frame}


\begin{frame}{Lesson 5}{Data Pipelines}
\LARGE
"Data Pipeline deals with information that is flowing from one end to 
another. In simple words, we can say collecting the data from various 
resources than processing it as per requirement and transferring it 
to the destination by following some sequential activities."\\~\\
Source: www.geeksforgeeks.org
\end{frame}

\begin{frame}[plain,noframenumbering]
\makebox[\linewidth]{\includegraphics[width=\paperwidth]{Images/pipeline1}}
\end{frame}


\begin{frame}{Lesson 5}{Data Pipelines}
\Huge
\begin{center}
AMAZON
\end{center}
\end{frame}


\begin{frame}{Lesson 5}{Data Pipelines}
\LARGE
"A data pipeline is a series of processing steps to prepare 
enterprise data for analysis. Organizations have a large volume of 
data from various sources like applications, Internet of Things (IoT)
devices, and other digital channels. However, raw data is useless; it
must be moved, sorted, filtered, reformatted, and analyzed for
business intelligence. A data pipeline includes various technologies
to verify, summarize, and find patterns in data to inform business 
decisions."
\end{frame}


\begin{frame}{Lesson 5}{Data Pipelines}
\LARGE
"Just like a water pipeline moves water from the reservoir to your
taps, a data pipeline moves data from the collection point to
storage. A data pipeline extracts data from a source, makes changes,
then saves it in a specific destination. We explain the critical
components of data pipeline architecture below."
\end{frame}


\begin{frame}[plain,noframenumbering]
\makebox[\linewidth]{\includegraphics[width=\paperwidth]{Images/amazon_pipeline}}
\end{frame}


\begin{frame}{Lesson 5}{Data Pipelines}
\Huge
 Data pipeline = a program or what? Lets rollback time... 
 \end{frame}


\begin{frame}{Lesson 5}{Data Pipelines}
\Huge
\begin{center}
1969. UNIX 
\end{center}
\end{frame}

\begin{frame}{Lesson 5}{Data Pipelines}
\LARGE
\textbf{The very first data pipelines}\\~\\
As early as 1964 Douglas McIlroy thought about a mechanism how we 
could connect programs same way like we connect and combine
garden hoses together. 
\end{frame}


\begin{frame}[plain,noframenumbering]
\makebox[\linewidth]{\includegraphics[width=\paperwidth]{Images/garden-pipeline}}
\end{frame}


\begin{frame}[plain,noframenumbering]
\makebox[\linewidth]{\includegraphics[width=\paperwidth]{Images/garden-pipeline2}}
\end{frame}


\begin{frame}[plain,noframenumbering]
\makebox[\linewidth]{\includegraphics[width=\paperwidth]{Images/garden-pipeline3}}
\end{frame}

\begin{frame}{Lesson 5}{Data Pipelines}
\LARGE
\textbf{Connecting Garden Hoses}\\~\\
\begin{itemize}
    \item it should connect different hoses
    \item as efficiently as possible
    \item with no leaks
    \item covering all the garden
\end{itemize}
\end{frame}



\begin{frame}{Lesson 5}{Data Pipelines}
\LARGE
\textbf{Pipelines are everywhere}\\~\\
\begin{itemize}
    \item transport fresh water pipelines
    \item collection pipelines from ground gas and oil to refineries
    \item wastewater pipelines
    \item distribution pipelines
\end{itemize}
\end{frame}


\begin{frame}[plain,noframenumbering]
\makebox[\linewidth]{\includegraphics[width=\paperwidth]{Images/water-pipeline}}
\end{frame}



\begin{frame}{Lesson 5}{Data Pipelines}
\Huge
 The idea was to reuse the model from other industries to CS
 \end{frame}


\begin{frame}
\begin{center}
\Huge
\begin{quote}
\textbf{"A pipeline is a mechanism for connecting the output of one program directly and conveniently into the 
    input of another program."}
\begin{flushright}
{--- Brian Kernighan, UNIX father}	
\end{flushright}
\end{quote}
\end{center}
\end{frame}


\begin{frame}{Lesson 5}{Data Pipelines}
\Huge
 The very first models of data pipelines were defined, introduced and 
 implemented in the operating systems. \textbf{Enter UNIX}
 \end{frame}




\begin{frame}{Lesson 5}{Data Pipelines}
\LARGE
\textbf{UNIX pipelines}\\~\\
The very first data pipeline was implemented for UNIX based operating
systems. The ideas was very simple: combine one or many programs 
using the \textbf{pipe} operator \text{\textbar}
\end{frame}


\begin{frame}{Lesson 5}{Data Pipelines}
\Huge
\begin{center}
\textbf{program} \text{\textbar} \textbf{program} \text{\textbar} ... 
\end{center}
\end{frame}



\begin{frame}[plain,noframenumbering]
\makebox[\linewidth]{\includegraphics[width=\paperwidth]{Images/pipeline}}
\end{frame}


\begin{frame}{Lesson 5}{Data Pipelines}
\Huge
The pipeline: is connecting one or many programs together! It is not 
a single monolithic program.
\end{frame}


\begin{frame}{Lesson 5}{Data Pipelines}
\LARGE
\textbf{UNIX pipelines}\\~\\
\Huge
But why connect different programs instead of having a single big 
program to handle all the input data?
\end{frame}


\begin{frame}{Lesson 5}{Data Pipelines}
\Huge
\begin{center}
Few reasons behind... have small, \textbf{modular} programs which 
can connect together to solve a problem quickly and efficiently.
\end{center}
\end{frame}

\begin{frame}{Lesson 5}{Data Pipelines}
\Huge
\begin{center}
By \alert{enforcing} a strict \alert{order} of execution!!!
\end{center}
\end{frame}


\begin{frame}{Lesson 5}{Data Pipelines}
\huge
\begin{itemize}
    \item smaller programs are easy to develop 
    \item and are simple and cheaper to maintain
    \item as well the amount of data might not fit one program
    \item no temporary files or data in between
\end{itemize}
\end{frame}


\begin{frame}[plain,noframenumbering]
\makebox[\linewidth]{\includegraphics[width=\paperwidth]{Images/unix-pipeline}}
\end{frame}



\begin{frame}{Lesson 5}{Data Pipelines}
\LARGE
\textbf{UNIX pipelines}\\~\\
Connects different UNIX programs: like ls, sort, wc, grep to solve a problem using the \text{\textbar} pipe character
\end{frame}


\begin{frame}[plain,noframenumbering]
\makebox[\linewidth]{\includegraphics[width=\paperwidth]{Images/unix-pipeline2}}
\end{frame}


\begin{frame}[fragile]
\LARGE
\textbf{UNIX pipelines, example}\\~\\
\begin{lstlisting}[language=sh]
$ ls -lrt Images/*.png | wc -l
     140
\end{lstlisting}
\end{frame}


\begin{frame}[fragile]
\LARGE
\textbf{Sort all images which were produced during Feb, example}\\
\Large
\begin{lstlisting}[language=sh]
$ ls -l Images/ | awk '{print $6,$7,$9}' | grep Feb | sort
Feb 26 treevsgraph.png
Feb 26 treevsgraph2.png
Feb 26 tsp.png
Feb 3 Alan_turing.jpg
Feb 3 Bombe.jpg
Feb 4 Queue.png
...
Feb 4 factorial2.png
Feb 4 recursion.png
\end{lstlisting}
\end{frame}



\begin{frame}[fragile]
\LARGE
\textbf{Sort all images which were produced during Feb, example}\\
\Large
\begin{lstlisting}[language=sh]
$ ls -l Images/ | awk '{print $6,$7,$9}' | grep Feb | sort -V
Feb 19 hash_function4.png
Feb 19 hash_function5.png
Feb 25 binary-tree.png
Feb 26 btree.png
Feb 26 depth-height.png
Feb 26 directed-graph.png
Feb 26 graph.png
Feb 26 selfbalanced-tree.png
Feb 26 treevsgraph.png
...
\end{lstlisting}
\end{frame}


\begin{frame}{Lesson 5}{Data Pipelines}
\Huge
\begin{center}
But how about the pipeline's \alert{speed} and overall
\alert{performance}? Should we care about?
\end{center}
\end{frame}



\begin{frame}[fragile]
\LARGE
\textbf{Search for all png images measuring the elapsed time, example}\\
\Large
\begin{lstlisting}[language=sh]
$ time find $HOME -type f | grep png
...
/myhome/feynman_path/examples/no-entanglement.png
/myhome/feynman_path/examples/no-interference-circuit.png
/myhome/feynman_path/examples/entanglement.png
/myhome/feynman_path/examples/no-entanglement-circuit.png

real 0m9.167s
user 0m0.450s
sys 0m3.147s
\end{lstlisting}
\end{frame}



\begin{frame}[fragile]
\LARGE
\textbf{Search  and sort for all jpg and png images measuring the elapsed time, example}\\
\Large
\begin{lstlisting}[language=sh]
$ time find $HOME -type f | egrep 'png|jpg' | sort -V
...
/myhome/.vscode/extensions/osi-certified-72x60.png
/myhome/.vscode/extensions/osi-certified-72x60.png
/myhome/.vscode/extensions/osi-certified-72x60.png
/myhome/.vscode/extensions/logo.png
/myhome/.vscode/extensions/lldb.png

real 0m9.240s
user 0m0.773s
sys 0m3.034s
\end{lstlisting}
\end{frame}


\begin{frame}[fragile]
\LARGE
\textbf{Search and sort for all jpg, png measuring the elapsed time under Windows, example}\\
\Large
\begin{lstlisting}[language=sh]
$ time find /myhome -type f | grep png
...
/myhome/.vscode/extensions/osi-certified-72x60.png
/myhome/.vscode/extensions/osi-certified-72x60.png
/myhome/.vscode/extensions/osi-certified-72x60.png
/myhome/.vscode/extensions/images/logo.png
/myhome/.vscode/extensions/images/lldb.png

real    2m20.786s
user    0m0.421s
sys     0m1.686s

\end{lstlisting}
\end{frame}


\begin{frame}{Lesson 5}{Data Pipelines}
\Huge
\begin{center}
Measuring pipeline elapsed time matters!
\end{center}
\end{frame}


\begin{frame}{Lesson 5}{Data Pipelines}
\LARGE
UNIX Pipelines
\begin{itemize}
    \item a chain link of different parts or modules (programs)
    \item each part, designed to do one thing but do it well (atomicity)
    \item pipeline must be simple to change and expand (modularity)
    \item it should perform in time (performance matters)
\end{itemize}
\end{frame}


\begin{frame}{Lesson 5}{Data Pipelines}
\Huge
\begin{center}
From OS pipelines were adopted everywhere else in IT. Remember ETL?
\end{center}
\end{frame}



\begin{frame}{Lesson 5}{Data Pipelines}
\Huge
\begin{center}
From OS pipelines were adopted everywhere else in IT. Remember \alert{ETL}?
\end{center}
\end{frame}



\begin{frame}{Lesson 5}{Data Pipelines}
\LARGE
\textbf{ETL - Extract, transform, load}\\~\\
A three-phase computing process where data is extracted from an input source, 
transformed (including cleaning), and loaded somewhere else for further 
processing or analysis.
\end{frame}


\begin{frame}{Lesson 5}{Data Pipelines}
\LARGE
\textbf{ETL Systems}\\~\\
In large IT organisations, \alert{ETL systems} started to be deployed as 
in-house or 3rd party based applications, most common close to \alert{date 
warehousing}. 
\end{frame}



\begin{frame}{Lesson 5}{Data Pipelines}
\Huge
\begin{center}
Data warehousing?
\end{center}
\end{frame}


\begin{frame}{Lesson 5}{Data Pipelines}
\LARGE
\textbf{Data warehousing}\\~\\
"It is a system used for reporting and data analysis, a core component of 
business intelligence for any IT organization. Data warehouses are central 
repositories of data integrated from disparate sources. They store current and 
historical data organized in a way that is optimized for data analysis, 
generation of reports, and developing insights across the integrated data."
(wikipedia)
\end{frame}


\begin{frame}{Lesson 2}{}
\begin{center}
	\includegraphics[scale=0.12]{Images/data_warehouse}
\end{center}
Source: wikipedia
\end{frame}



\begin{frame}{Lesson 5}{Data Pipelines}
\LARGE
\textbf{Data warehousing}\\~\\
Data warehousing = it is a huge repository of different data sources.
But how is this different from a classic transactional database? Whats the 
difference?
\end{frame}



\begin{frame}{Lesson 5}{Data Pipelines}
\LARGE
\textbf{Transactional Databases}\\
\begin{itemize}
	\item OLTP (Online transaction processing)
	\item transaction-oriented applications
	\item Used to update data in real-time
	\item designed for write and read operations
	\item normalized data structure
	\item SQL and noSQL implementation
	\item high transaction data, multi-users accessing the system
\end{itemize}
\end{frame}


\begin{frame}{Lesson 5}{Data Pipelines}
\LARGE
\textbf{OLTP. Normalized data}\\~\\
Data in OLTP systems is typically normalized to reduce redundancy and improve 
data integrity. It is a technique for creating database tables with suitable 
columns and keys by decomposing a large table into smaller logical units.
\end{frame}


\begin{frame}{Lesson 5}{Data Pipelines}
\LARGE
\textbf{Data Warehousing}\\
\begin{itemize}
	\item large repository of different datasets
	\item historical data storage
	\item designed for complex read operations
	\item Data is denormalized to improve query performance and simplify data retrieval
	\item ETL mechanisms to consolidate data from multiple sources into a single central repository.
\end{itemize}
\end{frame}



\begin{frame}{Lesson 5}{Data Pipelines}
\LARGE
\textbf{Pipelines and ETL}\\~\\
Pipelines started to be used for different ETL systems and when deploying data 
warehousing systems. Fetching or capturing data, transforming and manipulating
data for a specific business reason all parts of a data pipelines.
\end{frame}


\begin{frame}{Lesson 5}{Data Pipelines}
\LARGE
\textbf{Data pipelines vs ETL pipelines}\\~\\
Data pipelines and ETL pipelines or ETL systems are sometimes used. There are
however some few and important differences between ETL and Data pipelines. 
\end{frame}


\begin{frame}{Lesson 5}{Data Pipelines}
\LARGE
\textbf{ETL vs Data Pipelines}\\
\begin{itemize}
	\item specific, clear structure and order
	\item Extract, Transform and Load
	\item not all data pipelines follow this order
	\item ETL is very much know for batch processing
	\item sometimes data pipelines do not transform data
\end{itemize}
\end{frame}


\begin{frame}{Lesson 5}{Data Pipelines}
\Huge
\begin{center}
Types of Data Pipelines
\end{center}
\end{frame}


\begin{frame}{Lesson 5}{Data Pipelines}
\LARGE
\textbf{Types}\\~\\
Like waterpipes, there are different type of data pipelines. The model of a 
basic OS pipelines went beyond operating systems. But still the very same 
principles apply no matter of pipeline type and complexity. (elapsed time, 
modularity, simplicity, atomicity)
\end{frame}


\begin{frame}{Lesson 5}{Data Pipelines}
\LARGE
Data Pipeline Types
\begin{itemize}
    \item batch - run at specific time, with large amounts of data at once
    \item streaming - processes the data in real-time
    \item data integration - merge data from different sources into a single central place
\end{itemize}
\end{frame}




\begin{frame}{Lesson 5}{Data Pipelines}
\Huge
\begin{center}
Data Capture (Ingestion), Transformation and Analysis
\end{center}
\end{frame}


\begin{frame}{Lesson 5}{Data Pipelines}
\LARGE
\textbf{Capturing Data}\\~\\
"Data is collected from various sources—including software-as-a-service (SaaS) 
platforms, internet-of-things (IoT) devices and mobile devices—and various data 
structures, both structured and unstructured data. Within streaming data, these 
raw data sources are typically known as producers, publishers, or senders." -
IBM 
\end{frame}


\begin{frame}{Lesson 5}{Data Pipelines}
\LARGE
\textbf{Capturing Data}\\~\\
\begin{itemize}
    \item from multiple data sources
    \item stored as raw data: flat files, on on a central repository
    \item no transformations, no changes to original data
\end{itemize}
\end{frame}


\begin{frame}{Lesson 5}{Data Pipelines}
\LARGE
\textbf{Data Ingestion}\\~\\
Data capturing or data collection is also known as data ingestion. Todays every 
IT organization uses data ingestion as the term for capturing data from one or 
many data sources. 
\end{frame}



\begin{frame}{Lesson 5}{Data Pipelines}
\LARGE
\textbf{Data Transformation}\\~\\
This turns raw data which was previously collected into a format required by 
the destination data repository. A number of processes might be applied to 
clean or format data in a specific way.
\end{frame}



\begin{frame}{Lesson 5}{Data Pipelines}
\LARGE
\textbf{Data Transformation, Example 1}\\~\\
A data stream may be ingested as a nested JSON format, and the data 
transformation stage will aim to flatten that JSON and extract the
key fields for analysis.
\end{frame}


\begin{frame}{Lesson 5}{Data Pipelines}
\LARGE
\textbf{Data Transformation, Example 2}\\~\\
Another data stream might be ingested as a XML file. We need to 
transform the XML input file into a SQL table to be stored in a SQL
database.
\end{frame}



\begin{frame}{Lesson 5}{Data Pipelines}
\LARGE
\textbf{Data Storage and Analysis}\\~\\
"The transformed data is then stored within a data repository, where
it can be exposed to various stakeholders. This transformed data are
typically known as consumers, subscribers, or recipients." (IBM)
\end{frame}



\begin{frame}{Lesson 5}{Data Pipelines}
\Huge
\begin{center}
Raw Data
\end{center}
\end{frame}


\begin{frame}{Lesson 5}{Data Pipelines}
\LARGE
\textbf{Raw Data}\\~\\
Raw data, also known as primary data, are data (e.g., numbers, 
instrument readings, figures, etc.) collected from a source. In the 
context of examinations, the raw data might be described as a raw 
score (after test scores)." (wikipedia)
\end{frame}



\begin{frame}{Lesson 5}{Data Pipelines}
\LARGE
\textbf{Raw Data}\\~\\
Raw data is fundamental to any data analysis.\\~\\
The primary, original data collected from a source, which has not
been transformed, aggregated or changed in any way.
\end{frame}



\begin{frame}{Lesson 5}{Data Pipelines}
\Huge
\begin{center}
The raw data is primary. The interface secondary.
\end{center}
\end{frame}



\begin{frame}{Lesson 5}{Data Pipelines}
\LARGE
\textbf{Time series}\\~\\
All recorded observations, metrics are variable measured
sequentially in time, called time series, stored as raw data. Raw
data is produced by a recorder, which fetches data from a system,
device or sensor, data which has not been modified or changed in any
way. The data is saved as CSV flat files on disk.
\end{frame}



\begin{frame}{Lesson 5}{Data Pipelines}
\LARGE
\textbf{Universal Format}\\~\\
The original recorded data, available as ascii text, CSV files or 
plain old sql dump files will always work with any software, user-
interface or applications. The raw data never becomes obsolete, old 
or dated by any technology standards or new software methodologies.
\end{frame}



\begin{frame}{Lesson 5}{Data Pipelines}
\LARGE
\textbf{Data Centric}\\~\\
Many software applications are putting effort in providing shiny and 
modern front-ends. This reflect an interface centric, rather than 
data centric approach. Any UI will soon become obsolete. Effort 
should be put in capturing and consolidating raw data. The data is 
primary, the user-interface secondary.
\end{frame}



\begin{frame}{Lesson 5}{Data Pipelines}
\LARGE
\textbf{Open Data}\\~\\
A data centric system will always be able to process and offer 
original raw data in a simple manner for usage, without constraints. 
The data can be made available directly for download or accessible 
over a programmable interface which other programs and applications 
can use.
\end{frame}




\begin{frame}{Lesson 5}{Data Pipelines}
\Huge
\begin{center}
Not all data is good. How do you know to capture or ingest the right 
data? 
\end{center}
\end{frame}


\begin{frame}{Lesson 5}{Data Pipelines}
\LARGE
\textbf{Open Data}\\~\\
A data centric system will always be able to process and offer 
original raw data in a simple manner for usage, without constraints. 
The data can be made available directly for download or accessible 
over a programmable interface which other programs and applications 
can use.
\end{frame}


\begin{frame}{Lesson 5}{Data Pipelines}
\Huge
\begin{center}
Data aggregation
\end{center}
\end{frame}



\begin{frame}{Lesson 5}{Data Pipelines}
\LARGE
\textbf{Aggregating Data}\\~\\
"Data aggregation is the process of collecting and summarizing 
information from various databases to create combined datasets for 
analysis. It helps in organizing large amounts of data into a more 
usable format, making it easier to derive insights and make 
decisions." (wikipedia)
\end{frame}



\begin{frame}{Lesson 5}{Data Pipelines}
\Huge
\begin{center}
Pipeline efficiency
\end{center}
\end{frame}




\begin{frame}{Lesson 5}{Data Pipelines}
\LARGE
\textbf{How do you measure good or bad?}\\~\\
\begin{itemize}
    \item performance metrics
    \item data quality output
    \item errors
\end{itemize}
\end{frame}




\begin{frame}{Lesson 5}{Data Pipelines}
\LARGE
\textbf{Performance metrics}\\~\\
\begin{itemize}
    \item throughput
    \item latency (wait time)
    \item response time (wait time + service time)
    \item busy time
\end{itemize}
\end{frame}



\begin{frame}{Lesson 5}{Data Pipelines}
\LARGE
\textbf{Quality metrics}\\~\\
\begin{itemize}
    \item do we have the right output?
    \item is the output the right format
    \item is data consistent? 
    \item can we aggregate from this data?
    \item ensures that data remains consistent across different parts of the system or over time.
\end{itemize}
\end{frame}



\end{document}

